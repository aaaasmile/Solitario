\setuplayout[topspace=0.5in, backspace=1in, header=24pt, footer=36pt,
  height=middle, width=middle]
% uncomment the next line to see the layout
%\showframe
\setuppagenumbering[location={header,right}]

\setupbodyfont[pagella]

% Lingua
\setbreakpoints[compound] % break at / and -
\setuplanguage[it][
  goodies={lang-it.llg}, % prevent wrong ligatures
  lefthyphenmin=2,
  righthyphenmin=2,
  leftquotation=»,
  rightquotation=«,
  leftquote=›,
  rightquote=‹,
]
\mainlanguage[it]

\definetyping[powsh][bodyfont=small]
\definetyping[sh][bodyfont=small]

%urls - i primi comandi servoro per poter usare l'url
\setupinteraction [state=start]
\setupcolors      [state=start]
\setupurl         [color=blue]
\setupinteraction [state=start,color=blue]
\useURL[uri:Invido]
              [https://invido.it/progetti/solitario_progetto.html]
              []
              [Invido.it]
\useURL[uri:Klondike]
              [https://it.wikipedia.org/wiki/Klondike_(solitario)]
              []
              [Klondike]
\useURL[uri:GithubSolitario]
              [https://github.com/aaaasmile/Solitario]
              []
              [GithubSolitario]


\starttext

\title{Solitario 2.0}

Questo è il classico gioco del Solitario con le carte da Briscola.
Il gioco si ispira all'originale \from[uri:Klondike] come mostrato in \in{figura}[fig:ingame].
Il tavolo di gioco si suddivide in tre sezioni: il basamento degli Assi, le pile di carte e il mazzo delle
carte restanti.

\placefigure
    [][fig:ingame]
    {Inizio della partita}
    {\externalfigure[ingame.png][width=0.9\makeupwidth]}

\subject{Comandi di gioco}
Il Solitario si gioca con il mouse. Con la tastiera, invece, è possibile inviare dei comandi per:
\startitemize[1]
\item Iniziare usa nuova partita (tasto {\bold n})
\item Uscire dal gioco (tasto {\bold Escape})
\item Mostrare un'animazione  (tasto {\bold a})
\stopitemize
Il tasto destro del mouse viene usato su una carta scoperta per essere mandata di colpo
e automaticamente sul basamento degli Assi. Il tasto sinistro del mouse server per spostare le carte da una pila all'altra.

\subject{Scopo del gioco}
Il gioco termina, e il solitario risulta riuscito, quando le quattro pile in alto a destra, inizialmente vuote,
contengono tutte le carte oridinate dei quattro semi partendo dall'Asso fino al Re (vedi \in{figura}[fig:finepartita]).

\placefigure
    [][fig:finepartita]
    {Fine della partita}
    {\externalfigure[fine.png][width=0.9\makeupwidth]}

\subject{Movimenti delle carte}
Una carta può  essere spostata da una pila all'altra cliccando una carta scoperta e trasportandola sopra
un'altra carta scoperta di un'altra pila (Drag & Drop). La carta deve essere compatibile con la pila di destinazione.
Per esempio, una coppa può essere sovrapposta solo ad una carta che è o spade o bastoni.
Lo stesso vale per denari. I bastoni e le spade possono, invece, essere spostati solo sopra un denari o coppe.
Così una spada non può essere messa sopra un bastoni o un'altra spada. Lo stesso vale per bastoni.
Il denari non può essere messo sopra una coppa e neanche un altro denari. Lo stesso vale per coppe.

Non solo l'alternanza del seme è importante ma anche l'ordine, che deve essere decrescente.

Esempio: sotto un Fante ci va un Sette, sotto un Re un Cavallo e così via fino al Due.
Lo spostamento delle carte constente di poter scoprire le carte coperte sulle varie pile.

Quando si scopre un Asso allora lo si può mettere su uno dei quattro basamenti in alto a destra
(per esempio con il tasto destro del mouse).
Poi si passa al Due e così via fino al Re (vedi \in{figura}[fig:assobase]).

\placefigure
    [][fig:assobase]
    {Asso sul basamento}
    {\externalfigure[basamento.png][width=0.9\makeupwidth]}

Quando lo spostamento delle carte tra le pile o sul basamento degli Assi non è più possibile,
allora si può cliccare il mazzo di carte restanti in alto a destra (vedi \in{figura}[fig:mazzo]).
Il mazzo delle carte restanti può essere usato a ripetizione fino a quando si svuota completamente.

\placefigure
    [][fig:mazzo]
    {Mazzo di aiuto}
    {\externalfigure[mazzo.png][width=0.9\makeupwidth]}

\subject{Informazioni}
Versione: 2.0\\
Sorgenti: \from[uri:GithubSolitario]\\
Licenza: GPL-2.0 \\
Libreria: SDL 2 \\
Web: \from[uri:Invido]
