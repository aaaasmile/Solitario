\setuplayout[topspace=0.5in, backspace=1in, header=24pt, footer=36pt,
  height=middle, width=middle]
% uncomment the next line to see the layout
%\showframe
\setuppagenumbering[location={header,right}]

\setupbodyfont[pagella]

% Lingua
\setbreakpoints[compound] % break at / and -
\setuplanguage[it][
  goodies={lang-it.llg}, % prevent wrong ligatures
  lefthyphenmin=2,
  righthyphenmin=2,
  leftquotation=»,
  rightquotation=«,
  leftquote=›,
  rightquote=‹,
]
\mainlanguage[it]

\definetyping[powsh][bodyfont=small]
\definetyping[sh][bodyfont=small]

%urls - i primi comandi servoro per poter usare l'url
\setupinteraction [state=start]
\setupcolors      [state=start]
\setupurl         [color=blue]
\setupinteraction [state=start,color=blue]
\useURL[uri:Invido]
              [https://invido.it/progetti/solitario_progetto.html]
              []
              [Invido.it]
\useURL[uri:Klondike]
              [https://it.wikipedia.org/wiki/Klondike_(solitario)]
              []
              [Klondike]
\useURL[uri:GithubSolitario]
              [https://github.com/aaaasmile/Solitario]
              []
              [GithubSolitario]


\starttext

\title{Solitario 2.0}

\placefigure
    [][fig:inizio2]
    {Schermata iniziale}
    {\externalfigure[inizio2.png][width=0.9\makeupwidth]}

Questo è il classico gioco del Solitario con le carte da Briscola.
Il gioco si ispira all'originale \from[uri:Klondike] come mostrato in \in{figura}[fig:ingame].
Il tavolo di gioco si suddivide in tre sezioni: il basamento degli Assi, le pile di carte e il mazzo delle
carte restanti.

\placefigure
    [][fig:ingame]
    {Inizio della partita}
    {\externalfigure[ingame.png][width=0.9\makeupwidth]}

\subject{Comandi di gioco}
Il Solitario si gioca con il mouse. Con la tastiera, invece, è possibile inviare dei comandi per:
\startitemize[1]
\item Iniziare usa nuova partita (tasto {\bold n})
\item Uscire dal gioco (tasto {\bold Escape})
\stopitemize
Il tasto destro del mouse viene usato su una carta scoperta per essere mandata di colpo
e automaticamente sul basamento degli Assi. Il tasto sinistro del mouse server per spostare le carte da una pila all'altra.

\subject{Scopo del gioco}
Il gioco termina, e il solitario risulta riuscito, quando le quattro pile in alto a destra, inizialmente vuote,
contengono tutte le carte oridinate dei quattro semi partendo dall'Asso fino al Re (vedi \in{figura}[fig:finepartita]).

\placefigure
    [][fig:finepartita]
    {Fine della partita}
    {\externalfigure[fine.png][width=0.9\makeupwidth]}

\subject{Movimenti delle carte}
Una carta può  essere spostata da una pila all'altra. Come? Cliccando una carta scoperta e trasportandola sopra
un'altra carta scoperta di un'altra pila (Drag & Drop).
La carta deve essere compatibile con la pila di destinazione.
Per esempio, una carta di coppe può essere sovrapposta solo ad una carta che è o spade o bastoni.
Lo stesso vale per denari.
I bastoni e le spade possono, invece, essere spostati solo sopra un denari o coppe.
Così una spada non può essere messa sopra un bastoni o un'altra spada. Lo stesso vale per bastoni.
Il denari non può essere messo sopra una coppa e neanche sopra un altro denari. Lo stesso vale per coppe.
Questo è il vincolo di colore, dove spade e bastoni sono i semi neri, mentre coppe e denari sono i semi rossi.
Un seme rosso va messo sopra un seme nero e viceversa.

L'ordine, oltre al colore,  è anch'esso un vincolo e deve essere decrescente.

{\italic Esempio}: sotto un Fante ci va un Sette, sotto un Re un Cavallo e così via fino al Due.

Col mazzo del Tarocco Piemontese, l'ordine è: Re (R), Donna (D), Cavallo (C), Fante (V), Dieci, Nove, Otto,
Sette, Sei, Cinque, Quattro, Tre, Due, Asso.

Lo spostamento delle carte constente di poter scoprire le carte coperte sulle varie pile (basta cliccare sul dorso coperto).

Solo un Re può essere spostato sui basamenti delle pile del tavolo di gioco.

Quando si scopre un Asso allora lo si può mettere su uno dei quattro basamenti in alto a destra
(per esempio con il tasto destro del mouse).
Poi si passa al Due e così via fino al Re (vedi \in{figura}[fig:assobase]).
Il gioco termina quando i quattro basamenti degli assi sono completi, oppure quando il giocatore rinuncia a
continuare la partita perché è bloccato e non riesce più a continuare.


\placefigure
    [][fig:assobase]
    {Asso sul basamento}
    {\externalfigure[basamento.png][width=0.9\makeupwidth]}

Quando lo spostamento delle carte tra le pile o sul basamento degli Assi non è più possibile,
allora si può cliccare il mazzo delle carte restanti in alto a destra (vedi \in{figura}[fig:mazzo]).
Il mazzo delle carte restanti può essere usato a ripetizione fino a quando si svuota completamente.
Però bisogna fare attenzione che il punteggio descresce, e di molto, ogni volta che il mazzo delle carte
di supporto viene rigirato.

\placefigure
    [][fig:mazzo]
    {Mazzo di aiuto}
    {\externalfigure[mazzo.png][width=0.9\makeupwidth]}

\subject{Punteggio}
Durante il gioco viene aggiornato un punteggio sia in positivo che in negativo.
Quando il solitario riesce,  il punteggio viene memorizzato nella graduatoria dei migliori dieci, sempre che sia un
punteggio appartenente a questa categoria.
Il menu {\italic High Score} visualizza la classifica attuale.

\subsubject{Punteggio virtuoso (positivo)}
\startitemize[2]
\item Ogni carta portata dal mazzo di supporto su una pila del tavolo da gioco: +45 punti
\item Ogni carta girata sul tavolo: +25 punti
\item Ogni carta portata sul basamento degli assi: +60 punti
\stopitemize

\subsubject{Punteggio penalizzante (negativo)}
\startitemize[2]
\item Ogni secondo che passa: -1 Punto
\item Ogni carta che dal basamento degli assi torna sul tavolo: -75 punti
\item Quando il mazzo di supporto viene rigirato: -175 punti
\stopitemize

\subsubject{Bonus}
Quando il solitario riesce viene aggiunto un bonus al punteggio finale:
\startformula
bonus = 2 * punteggio - (10 * tempoImpiegatoInSecondi)
\stopformula
\startformula
punteggioFinale = bonus + punteggio
\stopformula

\subject{Tarocco Piemontese}

\placefigure
    [][fig:inizio2]
    {Tarocco Piemontese}
    {\externalfigure[tarocco.png][width=0.9\makeupwidth]}

Tra i vari mazzi supportati dal Solitario vie è anche lo splendido mazzo del Tarocco Piemontese.
È un mazzo di 56 carte che consente, quando il solitario riesce, di conseguire un maggiore punteggio
rispetto al mazzo da 40 carte della briscola. Ecco una sua spiegazione in lingua inglese che a cura di
Domenico Starna che è anche il creatore del mazzo.

\subsubject{Piemontese Tarot, Tarocco Piemontese, Taròck.}
Piemontése pronounced pee-é-mon-té-sé, é as in é french.

 This Deck has 78 playing cards.
 The symbols of the french suits are added left over and right under.
 The latin names are added over.
 The four suits are translated in latin

\starttabulate[|l|l|l|]
\HL
\NC {\bf italian} \NC {\bf english} \NC {\bf latin}
\NC \NR
\HL
 \NC Spade         \NC (Swords)        \NC Spathae   Spatha, ae, f.       (in french suits Spades) \NR
 \NC Coppe        \NC (Cups)            \NC Calices     Calix, icis, m.      (in french suits Hearts)   \NR
 \NC Denari         \NC (Coins)           \NC Denarii     Denarius, ii, m.    (in french suits Diamonds) \NR
 \NC Bastoni       \NC (Batons)          \NC Bacula     Baculum, i, n.      (in french suits Clubs)  \NR
\HL
\stoptabulate

 In the old french, XVII Century, they are called Espées, Couppes, Deniers and Bastons.
 The other 22 cards are called Triomphes. In the modern french Tarot, Tarot  Nouveau
 du 1890, there are Piques, Coeurs, Carreau et Trèfles. The other 21 cards are
 Atouts, the Excuse is considered apart.
   It is possible to translate with different words. For example Coin with Denarius, Aureum,
 Nummus, Pecunia etcetera. I choosed  Denarii, because more similar to italian Denari
 and french Denier. Chitarrella is a nickname of an author of XVIII Century. He writes in
 latin the rules of 3 card games, Mediatore, Tressette and Scopone. He calls the Jack:
 octo mulier. In the Sicilian Cards the Jack is a woman. In 1866 a neapolitan poet and
 bookseller translates the rules in neapolitan dialect, with some little variations. Chitarrella
 calls the suits at the beginning of the rule of the Mediatore:
 Mucro, cuppa, nummus baculusque sunt ordines quattuor chartarum.
 An example of the card games rules from De Regulis Scoponis, Chitarrella, is
 the rule number 42:
 \startitemize
  \item  Qui memoriam non habet et continuae attentioni minime idoneus est scoponem
  relinquat et eat jocatum ad nuces.
  \item Who has no good memory and he is not able to continuous attention, he abandons
  the card game Scopone and he goes to play with little balls.
  In the Aces there are added latin mottos.
 \item  Quisque faber fortunae suae. (Appius Claudius Caecus)
                                                                        Every man is the artisan of his own fortune
 \item  Carpe diem. (Horatius, Carmina I, 11)                Pluck the day
    (...nec Babylonios temptaris numeros... sapias... fugerit invida aetas...)
    (...nor attempt the horoscopes... be wise... envious time has passed...)
 \item Gutta cavat lapidem. (Ovidius, Ex Ponto III,10)   The drop excavates the stone
\stopitemize

\subsubject{Deck Tarocco Piemontese}
Deck Tarocco Piemontese (78 cards), dimension 62x106 mm, in piemontese dialect Taròck.
 It is used in Torino and in other villages of the region Piemonte for playing Tarocchi (pronounced
 taròkki) games. Nowadays the Tarocco Piemontese remains the only model of 78 cards Tarot
 deck in Italy. The deck is similar to the other classic tarot decks, but the picture cards and the
 Trionfi are double headed, double faced, they are divided horizontally in two symmetric halves.
 The Fool, numbered zero, is catching a butterfly.
 The Trionfo Triumph 15, The Devil, has an ironic image: the devils pull faces, showing their tongue out.
 In the King of Batons, the baton is a mace, a club for representation as in the oldest tarot, not a cudgel.
 The Batons are straight. The lateral Swords, in the cards from 2 to 10 Swords, are curved as sabres.
 The Ace of Cups is a vase with the flowers.

\subsubject{History}
In the XV Century It was called the Game of the Triumphs, Ludus Triumphorum in latin. The name
 Trionfi, Triumphs, means: they triumph, they win the cards of the other suits. The Triumphs beat the
 remaining cards. The word trump, Trumpf in german, comes by trionfi, triumphs.
 In the XVI Century it was called the game of the Tarocchi. Tarot comes from the Italian word tarocco.
 The verb taroccare has many meanings, in the game it means to play, to reply with a tarocco,
 stronger card.
 With the meaning to dispute with an other, taroccare comes perhaps from the ancient italian word
 altarocare, this from altarcare, altercari in latin, in italian altercare.
 In modern italian the verb taroccare, adjective taroccato, word tarocco, means to falsify, to counterfeit.
 Example: a taroccato high-quality watch.
 The word Tarocchi is referred to the 22 Triumphs, sometimes also to all 78 cards together.
 In Torino the players call tarocco, tarocchi the 22 Triumphs, as trumps, but the Fool does not take.
 In Torino they call the Triumphs with its number, except The Fool called Folle, or Matto.
 A modern simple italian card game is the Briscola, the Briscole are the stronger cards.
 Trionfi, Triumphs, Tarocchi, Briscole, Atouts, Trumps have the same meaning, stronger cards.
 The order of the Triumphs was also different in the year 1500.
 The first Triumph is Il Bagatello, Il Bagatto. This word comes from bagattella or bagatella, thing not
 important. Bagatelliere means player of bagattelle, magician, conjurer.
 \starttabulate[|l|l|l|]
 \NS[3] {\italic After there are the civil authority and the religious authorithy } \NR
 \NC 2 \NC L' Imperatrice \NC The Empress \NR
 \NC 3 \NC L' Imperatore \NC The Emperor \NR
 \NC 4 \NC La Papessa \NC The Popess \NR
 \NC 5 \NC Il Papa \NC  The Pope \NR
 \NS[3]  {\italic After there are images about the human life } \NR
 \NC 6 \NC La Temperanza \NC The Temperance \NR
 \NC 7 \NC L' Amore \NC The Love \NR
\NC 8 \NC Il Carro Trionfale \NC The Triumphal Chariot \NR
\NC 9 \NC La Fortezza, La Forza \NC The Force \NR
\NC 10 \NC La Ruota della Fortuna \NC The Wheel of Fortune \NR
\NC 11 \NC Il Tempo \NC The Time \NR
\NC 12 \NC L' Impiccato \NC The Hanged, the human impotence \NR
\NC 13 \NC La Morte \NC The Death. \NR
 \NS[3]  {\italic After there is the other world, the afterlife} \NR
\NC 14 \NC the hell with Il Diavolo \NC The Devil \NR
\NC 15 \NC La Saetta, Il Fulmine \NC The Lightning \NR
\NC 16 \NC La Stella \NC The Star \NR
\NC 17 \NC La Luna \NC The Moon \NR
\NC 18 \NC Il Sole \NC The Sun \NR
\NC 19 \NC Il Giudizio, L' Angelo \NC The Judgement, The Angel \NR
\NC 20 \NC La Giustizia \NC The Justice \NR
\NC 21 \NC Il Mondo \NC The World \NR
\NC 22 \NC or without number Il Matto \NC The Fool \NR
 \stoptabulate

 \subsubject{Game of the Tarocco Piemontese}
Taròck, in Torino. Resumé. \\
They play Tarocchi for 4 players (Tarocchi in 4, or a 4). \\
 In the game the players are in partnership, 2 against 2. \\
 The dealer (mazziere, scartante) gives 19 cards to each player and
 21 to himself. He chooses 2 of them and put them apart, they are
 the Scarto, the Discard; no K, no Tarocchi. But it is possible to discard
 the Triumph 1 if it is the only triumph of the 21.
 There are no Bids (Contrats).
 The match (la partita) has 4 games (smazzata, 4 smazzate), so each
 player is the dealer 1 time.
 A complete session, a round, has 3 matches, so each player plays
 with each of the 3 others 1 time in team. In total a session has
 12 games (3 matches x 4 games).
 The long suits, Batons (Bastoni) and Swords (Spade) correspond to the
 black french suits, Clubs and Spades. They rank normally from higher
 to lower: \\
         {\italic  K, Q, C, J, 10, 9, 8, 7, 6, 5, 4, 3, 2, 1 (Asso).} \\
 The ranking is different from french tarot in the short suits, Coins (Denari)
 and Cups (Coppe), correspondents to the red french suits, Diamonds
 and Hearts. They rank, backwards in the numeral cards, from higher
 to lower: \\
           {\italic K, Q, C, J, 1 (Asso), 2, 3, 4, 5, 6, 7, 8, 9, 10.} \\
 In Torino (but not in Milano in the old tarot) the trump 20, the Angel, is
 stronger than the trump 21, the World.
 There are the same 2 rules as in french tarot:
 follow suit (fournir à la couleur, fournir la couleur demandé),
 cut with a trump (couper),
 but not the rule of to play with a higher trump (surcouper).
 The Fool (il Matto or il Folle) is used as the Excuse in the french tarot.
 It does not take and can not be taken. The player shows this card to the
 others and put it in his tricks pile. So he is not obliged to play a trick
 disadvantegeous for him. If the Fool is played first in a trick, the second
 player leads the suit. \\
 Scoring is based on counting the points for each trick.
 1 for each trick, 4 for The Fool , for Triumph 1 and for 20 (in Milan with
 old tarot for 21 as in french tarot) and for K, 3 for Q, 2 for C, 1 for J;
 Example 4 low cards are 1 point; 3 low cards and a K are 5 points;
 2 low cards, a Q and a C are 6 points (1 for the trick, 3 for Q, 2 for C).
 The Fool is counted apart 4 points; the 3 cards of the same trick, if they
 are low cards, value 1 point. \\
 Also the Scarto, discard, with 2 cards, if they are low cards, is counted
 1 point, as a trick. \\
 If the team of the dealer has taken no tricks, it loses also the Scarto.
 If the team of the player with the Fool has taken no tricks, it loses
 also the Fool. \\
 The total points are 72 in a game; 40 for the 16 picture cards, 12 for three
 Triumphs, 19 for the tricks and 1 for the Scarto. The average for the
 two teams is 36 points (72:2) for each team. The scoring of each team
 in a game is the difference between his points and 36, positive or negative.
 In the game with 3 players, each plays for himself, the Scarto has 3 cards.
 Total points in a game are 78 points because there are 25 tricks.

Note. The man with a hourglass, The Time, has become the man with a lamp, The Hermit.
The Lightning has become The Tower.

\subject{Informazioni sul programma}
Versione: 2.0\\
Sorgenti: \from[uri:GithubSolitario]\\
Licenza: GPL-2.0 \\
Libreria: SDL 2 \\
Web: \from[uri:Invido]
